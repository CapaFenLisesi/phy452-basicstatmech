%
% Copyright � 2013 Peeter Joot.  All Rights Reserved.
% Licenced as described in the file LICENSE under the root directory of this GIT repository.
%
%\input{../blogpost.tex}
%\renewcommand{\basename}{basicStatMechLecture16}
%\renewcommand{\dirname}{notes/phy452/}
%\newcommand{\keywords}{Statistical mechanics, PHY452H1S, Fermion, chemical potential, high temperature limit, low temperature limit, integral approximation to sum, thermal de Broglie wavelength, density, ideal gas, pressure, energy, volume, degeneracy pressure}
%\input{../peeter_prologue_print2.tex}
%
%\beginArtNoToc
%\generatetitle{PHY452H1S Basic Statistical Mechanics.  Lecture 16: Fermi gas.  Taught by Prof.\ Arun Paramekanti}
%\chapter{Fermi gas}
%\label{chap:basicStatMechLecture16}
%
%\section{Disclaimer}
%
%Peeter's lecture notes from class.  May not be entirely coherent.
%

\paragraph{Review}

Continuing a discussion of \citep{pathriastatistical} \S 8.1 content.

We found

\begin{dmath}\label{eqn:basicStatMechLecture16:20}
n_{\Bk} = \inv{e^{\beta(\epsilon_k - \mu)} + 1}
\end{dmath}

With no spin

\begin{dmath}\label{eqn:basicStatMechLecture16:40}
\int n_\Bk \times \frac{d^3 k}{(2\pi)^3} = \rho
\end{dmath}

%\cref{fig:lecture16:lecture16Fig1}.
\imageFigure{../../figures/phy452/lecture16Fig1}{Occupancy at low temperature limit}{fig:lecture16:lecture16Fig1}{0.3}
%\cref{fig:lecture16:lecture16Fig1b}.
\imageFigure{../../figures/phy452/lecture16Fig1b}{Volume integral over momentum up to Fermi energy limit}{fig:lecture16:lecture16Fig1b}{0.3}

\begin{dmath}\label{eqn:basicStatMechLecture16:480}
\epsilon_{\txtF} = \frac{\Hbar^2 k_{\txtF}^2}{2m}
\end{dmath}

gives

\begin{dmath}\label{eqn:basicStatMechLecture16:60}
k_{\txtF} = (6 \pi^2 \rho)^{1/3}
\end{dmath}

\begin{dmath}\label{eqn:basicStatMechLecture16:80}
\sum_\Bk n_\Bk = N
\end{dmath}

\begin{dmath}\label{eqn:basicStatMechLecture16:100}
\Bk = \frac{2\pi}{L}(n_x, n_y, n_z)
\end{dmath}

This is for periodic boundary conditions \footnote{I filled in details in the last lecture using a particle in a box, whereas this periodic condition was intended.  We see that both achieve the same result}, where

\begin{dmath}\label{eqn:basicStatMechLecture16:120}
\Psi(x + L) = \Psi(x)
\end{dmath}

\paragraph{Moving on}

\begin{dmath}\label{eqn:basicStatMechLecture16:140}
\sum_{k_x} n(\Bk) = \sum_{p_x} \Delta p_x n(\Bk)
\end{dmath}

with
\begin{dmath}\label{eqn:basicStatMechLecture16:160}
\Delta k_x = \frac{2 \pi}{L} \Delta p_x
\end{dmath}

this gives

\begin{dmath}\label{eqn:basicStatMechLecture16:180}
\sum_{k_x} n(\Bk)
= \sum_{n_x} \frac{L}{2\pi} \Delta k_x
\rightarrow
\frac{L}{2\pi} \int d k_x
\end{dmath}

Over all dimensions

\begin{dmath}\label{eqn:basicStatMechLecture16:200}
\sum_{\Bk} n_\Bk =
\lr{\frac{L}{2\pi}}^3
\lr{ \int d^3 \Bk}
n(\Bk)
=
N
\end{dmath}

so that

\begin{dmath}\label{eqn:basicStatMechLecture16:220}
\rho = \int \frac{d^3 \Bk}{(2 \pi)^3}
\end{dmath}

Again

\begin{dmath}\label{eqn:basicStatMechLecture16:240}
k_{\txtF} = (6 \pi^2 \rho)^{1/3}
\end{dmath}

\makeexample{Spin considerations}{example:basicStatMechLecture16:1}{

%FIXME:
%\begin{itemize}
%\item
%S = \inv{2} \rightarrow \mbox{\(2\) states at each \(\Bk\)}
%\item
%Spin \(S \rightarrow \mbox{(\)2 S + 1\() states at each \)\Bk\(, where \)m_s = -S, -S + 1, \cdots, S$.}
%\end{itemize}

\begin{dmath}\label{eqn:basicStatMechLecture16:260}
\sum_{\Bk, m_s} = N
\end{dmath}
\begin{dmath}\label{eqn:basicStatMechLecture16:280}
\sum_{\Bk, m_s} \inv{e^{\beta(\epsilon_k - \mu)} + 1}
= (2 S + 1)
\lr{ \int \frac{d^3 \Bk}{(2 \pi)^3} n(\Bk) } L^3
\end{dmath}

This gives us

\begin{dmath}\label{eqn:basicStatMechLecture16:300}
k_{\txtF} =
\lr{\frac{ 6 \pi^2 \rho }{2 S + 1}}^{1/3}
\end{dmath}

and again

\begin{dmath}\label{eqn:basicStatMechLecture16:320}
\epsilon_{\txtF} = \frac{\Hbar^2 \kF^2}{2m}
\end{dmath}
}

\paragraph{High Temperatures}

Now we want to look at the at higher temperature range, where the occupancy may look like \cref{fig:lecture16:lecture16Fig2}

\imageFigure{../../figures/phy452/lecture16Fig2}{Occupancy at higher temperatures}{fig:lecture16:lecture16Fig2}{0.3}

\begin{dmath}\label{eqn:basicStatMechLecture16:340}
\mu(T = 0) = \epsilon_{\txtF}
\end{dmath}
\begin{dmath}\label{eqn:basicStatMechLecture16:360}
\mu(T \rightarrow \infty) \rightarrow - \infty
\end{dmath}

so that for large \(T\) we have

\begin{dmath}\label{eqn:basicStatMechLecture16:380}
\inv{e^{\beta(\epsilon_k - \mu)} + 1} \rightarrow e^{-\beta(\epsilon_k - \mu)}
\end{dmath}

\begin{dmath}\label{eqn:basicStatMechLecture16:500}
\rho
= \int \frac{d^3 \Bk}{(2 \pi)^3} e^{\beta \mu} e^{-\beta \epsilon_k}
=
e^{\beta \mu}
\int \frac{d^3 \Bk}{(2 \pi)^3}
e^{-\beta \epsilon_k}
=
e^{\beta \mu}
\int dk \frac{4 \pi k^2}{(2 \pi)^3}
e^{-\beta \Hbar^2 k^2/2m}.
\end{dmath}

Mathematica (or integration by parts) tells us that

\begin{dmath}\label{eqn:basicStatMechLecture16:680}
\inv{(2 \pi)^3}
\int 4 \pi^2 k^2 dk
e^{-a k^2} = \inv{(4 \pi a )^{3/2}},
\end{dmath}

so we have

\begin{dmath}\label{eqn:basicStatMechLecture16:700}
\rho
=
e^{\beta \mu} \lr{ \frac{2m}{ 4 \pi \beta \Hbar^2} }^{3/2}
=
e^{\beta \mu} \lr{ \frac{2 m \kB T 4 \pi^2 }{ 4 \pi h^2} }^{3/2}
=
e^{\beta \mu} \lr{ \frac{2 m \kB T \pi }{ h^2} }^{3/2}
\end{dmath}

Introducing \(\lambda\) for the \underlineAndIndex{thermal de Broglie wavelength}, \(\lambda^3 \sim T^{-3/2}\)

\begin{dmath}\label{eqn:basicStatMechLecture16:400}
\lambda \equiv \frac{h}{\sqrt{2 \pi m \kB T}},
\end{dmath}

we have
\begin{dmath}\label{eqn:basicStatMechLecture16:720}
\rho = e^{\beta \mu}  \inv{\lambda^3}.
\end{dmath}

Does it make any sense to have density as a function of temperature?  An inappropriately extended to low temperatures plot of the density is found in \cref{fig:lecture16:lecture16Fig6} for a few arbitrarily chosen numerical values of the \textAndIndex{chemical potential} \(\mu\), where we see that it drops to zero with temperature.  I suppose that makes sense if we are not holding volume constant.

\imageFigure{../../figures/phy452/lecture16Fig6}{Density as a function of temperature}{fig:lecture16:lecture16Fig6}{0.3}

We can write

\boxedEquation{eqn:basicStatMechLecture16:420}{
e^{\beta \mu} = \lr{\rho \lambda^3}
}

\begin{dmath}\label{eqn:basicStatMechLecture16:440}
\frac{\mu}{\kB T} = \ln \lr{ \rho \lambda^3 } \sim -\frac{3}{2} \ln T
\end{dmath}

or (taking \(\rho\) (and/or volume?) as a constant) we have for large temperatures

\begin{dmath}\label{eqn:basicStatMechLecture16:460}
\mu \propto -T \ln T
\end{dmath}

The chemical potential is plotted in \cref{fig:lecture16:lecture16Fig3}, whereas this \(- \kB T \ln  \kB T\) function is plotted in \cref{fig:lecture16TLogTPlot:lecture16TLogTPlotFig7}.  The contributions to \(\mu\) from the \(\kB T \ln (\rho h^3 (2 \pi m)^{-3/2})\) term are dropped for the high temperature approximation.  %It's not entirely clear to me how we justify dropping the \(\ln \rho\), since we see that \(\rho = \rho(\mu, T)\) gets very small in the high temperature limit, but if all we are about is that \(\mu\) is large and negative, then this isn't inconsisent.

\imageFigure{../../figures/phy452/lecture16Fig3}{Chemical potential over degenerate to classical range}{fig:lecture16:lecture16Fig3}{0.3}
\imageFigure{../../figures/phy452/lecture16TLogTPlotFig7}{High temp approximation of chemical potential, extended back to \(T = 0\)}{fig:lecture16TLogTPlot:lecture16TLogTPlotFig7}{0.3}

\paragraph{Pressure}

\begin{dmath}\label{eqn:basicStatMechLecture16:520}
P = - \PD{V}{E}
\end{dmath}

For a classical ideal gas as in \cref{fig:lecture16:lecture16Fig4} we have

\imageFigure{../../figures/phy452/lecture16Fig4}{Ideal gas pressure vs volume}{fig:lecture16:lecture16Fig4}{0.3}

\begin{dmath}\label{eqn:basicStatMechLecture16:540}
P = \rho \kB T
\end{dmath}

For a Fermi gas at \(T = 0\) using the low temperature approximation \eqnref{eqn:basicStatMechLecture15:400} we have

\begin{dmath}\label{eqn:basicStatMechLecture16:560}
E
= \sum_\Bk \epsilon_k n_k
= \sum_\Bk \epsilon_k \Theta(\mu_0 - \epsilon_k)
= \frac{V}{(2\pi)^3} \int_{\epsilon_k < \mu_0} \frac{\Hbar^2 \Bk^2}{2 m} d^3 \Bk
= \frac{V}{(2\pi)^3} \int_0^{\kF} \frac{\Hbar^2 \Bk^2}{2 m} d^3 \Bk
= \frac{V}{(2\pi)^3}
\frac{\Hbar^2}{2 m}
\int_0^{\kF} k^2 4 \pi k^2 d k
\propto \kF^5
\end{dmath}

Specifically,

\begin{dmath}\label{eqn:basicStatMechLecture16:580}
E(T = 0) = V \times
%\lr{
\frac{3}{5}
\mathLabelBox{
\epsilon_{\txtF}
}{\(\sim \kF^2\)}
\mathLabelBox
[
   labelstyle={below of=m\themathLableNode, below of=m\themathLableNode}
]
{
\rho
}{\(\sim \kF^3\)}
%}
\end{dmath}

or

\begin{dmath}\label{eqn:basicStatMechLecture16:600}
\frac{E}{N} = \frac{3}{5} \epsilon_{\txtF}
\end{dmath}

\begin{dmath}\label{eqn:basicStatMechLecture16:620}
E
= \frac{3}{5} N \frac{\Hbar^2}{2 m} \lr{ 6 \pi^2 \frac{N}{V} }^{2/3}
= a V^{-2/3},
\end{dmath}

so that

\begin{dmath}\label{eqn:basicStatMechLecture16:640}
\PD{V}{E} = -\frac{2}{3} a V^{-5/3}.
\end{dmath}

\begin{dmath}\label{eqn:basicStatMechLecture16:660}
P
= -\PD{V}{E}
= \frac{2}{3} \lr{a V^{-2/3} } V^{-1}
= \frac{2}{3} \frac{E}{V}
=
\frac{2}{3} \lr{
\frac{3}{5} \epsilon_{\txtF} \rho
}
=
\frac{2}{5}
\epsilon_{\txtF} \rho.
\end{dmath}

We see that the pressure ends up deviating from the classical result at low temperatures, as sketched in \cref{fig:lecture16:lecture16Fig5}.  This low temperature limit for the pressure \(2 \epsilon_{\txtF} \rho/5\) is called the \underlineAndIndex{degeneracy pressure}.

\imageFigure{../../figures/phy452/lecture16Fig5}{Fermi degeneracy pressure}{fig:lecture16:lecture16Fig5}{0.3}

%\EndArticle
