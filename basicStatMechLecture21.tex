%
% Copyright � 2013 Peeter Joot.  All Rights Reserved.
% Licenced as described in the file LICENSE under the root directory of this GIT repository.
%
%\input{../blogpost.tex}
%\renewcommand{\basename}{basicStatMechLecture21}
%\renewcommand{\dirname}{notes/phy452/}
%\newcommand{\keywords}{Statistical mechanics, PHY452H1S, phonon modes, harmonic oscillator}
%\input{../peeter_prologue_print2.tex}
%
%\beginArtNoToc
%\generatetitle{PHY452H1S Basic Statistical Mechanics.  Lecture 21: Phonon modes.  Taught by Prof.\ Arun Paramekanti}
%%\chapter{Phonon modes}
\label{chap:basicStatMechLecture21}
%
%\section{Disclaimer}
%
%Peeter's lecture notes from class.  May not be entirely coherent.
%
%These are notes for the last class, which included a lot of discussion not captured by this short set of notes (as well as slides which were not transcribed).
%
%\paragraph{Phonons and other systems}

We have phenomena in matter that are very similar to Boson particle statistics.  We can discuss lattice vibrations in a solid.  These are called phonon modes, and will have the same distribution function where the only difference is that the speed of light is replaced by the speed of the sound wave in the solid.  Once we understand the photon system, we are able to look at other Bose distributions such as these phonon systems.  %We'll touch on this very briefly next time.

If we model a solid as a set of interconnected springs, as in \cref{fig:lecture21:lecture21Fig1}, then the potentials are of the form

\imageFigure{../figures/phy452-basicstatmech/lecture21Fig1}{Solid oscillator model}{fig:lecture21:lecture21Fig1}{0.15}

\begin{equation}\label{eqn:basicStatMechLecture21:20}
V = \inv{2} C \sum_n
\lr{u_n - u_{n+1}}^2,
\end{equation}

with kinetic energies

\begin{equation}\label{eqn:basicStatMechLecture21:40}
K = \sum_n \frac{p_n^2}{2m}.
\end{equation}

It's possible to introduce generalized forces

\begin{equation}\label{eqn:basicStatMechLecture21:60}
F = -\PD{u_n}{V}
\end{equation}

Can differentiate

\begin{equation}\label{eqn:basicStatMechLecture21:80}
m \frac{d^2 u_n}{dt^2} =
- C \lr{ u_n - u_{n+1}}
- C \lr{ u_n - u_{n-1}}
\end{equation}

Assuming a Fourier representation

\begin{equation}\label{eqn:basicStatMechLecture21:100}
u_n = \sum_k \tilde{u}(k) e^{i k n a},
\end{equation}

we find

\begin{equation}\label{eqn:basicStatMechLecture21:120}
m \frac{d^2 \tilde{u}(k)}{dt^2} = - 2 C
\lr{ 1 - \cos k a}
\tilde{u}(k)
\end{equation}

This looks like a harmonic oscillator with

\begin{equation}\label{eqn:basicStatMechLecture21:140}
\omega(k) = \sqrt{ \frac{2 C}{m} ( 1 - \cos k a)}.
\end{equation}

This is plotted in \cref{fig:lecture21:lecture21Fig2}.  In particular note that for for \(k a \ll 1\) we can use a linear approximation

\imageFigure{../figures/phy452-basicstatmech/lecture21Fig2}{Angular frequency of solid oscillator model.}{fig:lecture21:lecture21Fig2}{0.3}

\begin{equation}\label{eqn:basicStatMechLecture21:160}
\omega(k) \approx \sqrt{ \frac{C}{m} a^2 } \Abs{k}.
\end{equation}

Experimentally, looking at specific for a complex atomic structure like Gold, we find for example good fit for a model such as

\begin{equation}\label{eqn:basicStatMechLecture21:180}
C
\sim
\mathLabelBox{A T}{Contribution due to electrons}
+
\mathLabelBox
[
   labelstyle={below of=m\themathLableNode, below of=m\themathLableNode}
]
{B T^3.}{Contribution due to phonon like modes}
\end{equation}

where the phonon like modes are associated with linear energy-momentum relationships.

%\EndNoBibArticle
