%
% Copyright � 2013 Peeter Joot.  All Rights Reserved.
% Licenced as described in the file LICENSE under the root directory of this GIT repository.
%
\makeoproblem{Generating function}{basicStatMech:problemSet2:2}{2013 problem set 2, problem 2}{

The Fourier transform of the probability distribution defined above \(\tilde{P}(k)\) is called the ``generating function'' of the distribution. Show that the \(n\)-th derivative of this generating function \(\partial^n \tilde{P}(k)/\partial k^n\) at the origin \(k = 0\) is related to the \(n\)-th moment of the distribution function defined via \(\expectation{x^n} = \int dx P(x) x^n\). We will later see that the ``partition function'' in statistical mechanics is closely related to this concept of a generating function, and derivatives of this partition function can be related to thermodynamic averages of various observables.
} % makeoproblem

\makeanswer{basicStatMech:problemSet2:2}{

\begin{dmath}\label{eqn:basicStatMechProblemSet2Problem2:10}
\evalbar{
\frac{\partial^n}{\partial k^n}
   \tilde{P}(k)
}{k = 0}
=
\evalbar{
\frac{\partial^n}{\partial k^n}
\left(
\int_{-\infty}^\infty dx P(x) \exp\left( -i k x \right)
\right)
}{k = 0}
=
\evalbar{
\left(
\int_{-\infty}^\infty dx P(x) (-i x)^n
\exp\left( -i k x \right)
\right)
}{k = 0}
=
(-i)^n \int_{-\infty}^\infty dx P(x) x^n
= (-i)^n \expectation{x^n}
\end{dmath}

%For even and odd \(n\) respectively, this is
%
%\begin{subequations}
%\begin{equation}\label{eqn:basicStatMechProblemSet2Problem2:30}
%\evalbar{
%\frac{\partial^n}{\partial k^n}
%   \tilde{P}(k)
%}{k = 0}
%= (-1)^{n/2} \expectation{x^n},
%\end{equation}
%\begin{equation}\label{eqn:basicStatMechProblemSet2Problem2:50}
%\evalbar{
%\frac{\partial^n}{\partial k^n}
%   \tilde{P}(k)
%}{k = 0}
%= i (-1)^{n} \expectation{x^n}.
%\end{equation}
%\end{subequations}
}
