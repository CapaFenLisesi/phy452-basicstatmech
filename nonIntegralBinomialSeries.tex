%
% Copyright � 2013 Peeter Joot.  All Rights Reserved.
% Licenced as described in the file LICENSE under the root directory of this GIT repository.
%
%\input{../blogpost.tex}
%\renewcommand{\basename}{nonIntegralBinomialSeries}
%\renewcommand{\dirname}{notes/phy452/}
%\newcommand{\keywords}{binomial coefficient, factorial, gamma function, Taylor expansion, binomial series}
%
%\input{../peeter_prologue_print2.tex}
%
%\beginArtNoToc
%
%\generatetitle{Non integral binomial coefficient}
\label{chap:nonIntegralBinomialSeries}

In \citep{pathriastatistical} appendix \S F was the use of binomial coefficients in a non-integral binomial expansion.  This surprised me, since I'd never seen that before.  However, on reflection, this is a very sensible notation, provided the binomial coefficients are defined in terms of the gamma function.  Let's explore this little detail explicitly.

\paragraph{Taylor series}

We start with a Taylor expansion of

\begin{dmath}\label{eqn:nonIntegralBinomialSeries:20}
f(x) = (a + x)^b.
\end{dmath}

Our derivatives are

\begin{equation}\label{eqn:nonIntegralBinomialSeries:40}
\begin{aligned}
f'(x) &= b (a + x)^{b-1} \\
f''(x) &= b(b-1) (a + x)^{b-2} \\
f^3(x) &= b(b-1)(b-(3-1)) (a + x)^{b-3} \\
\vdots & \\
f^k(x) &= b(b-1)\cdots (b-(k-1)) (a + x)^{b-k}.
\end{aligned}
\end{equation}

Our Taylor series is then

\begin{dmath}\label{eqn:nonIntegralBinomialSeries:60}
(a + x)^b = \sum_{k = 0}^\infty \inv{k!} b(b-1)\cdots (b-(k-1)) a^{b-k} x^k.
\end{dmath}

Note that if \(b\) is a positive integer, then all the elements of this series become zero at \(b = k - 1\), or

\begin{dmath}\label{eqn:nonIntegralBinomialSeries:80}
(a + x)^b = \sum_{k = 0}^k \inv{k!} b(b-1)\cdots (b-(k-1)) a^{b-k} x^k.
\end{dmath}

\paragraph{Gamma function}

Let's now relate this to the gamma function.  From \citep{abramowitz1964handbook} \S 6.1.1 we have

\begin{dmath}\label{eqn:nonIntegralBinomialSeries:100}
\Gamma(z) = \int_0^\infty t^{z-1} e^{-t} dt.
\end{dmath}

Iteratively integrating by parts, we find the usual relation between gamma functions of integral separation

\begin{dmath}\label{eqn:nonIntegralBinomialSeries:120}
\Gamma(z + 1)
= \int_0^\infty t^{z} e^{-t} dt
= \int_0^\infty t^{z} d \left( \frac{ e^{-t}}{-1} \right)
=
\evalrange{
t^z \frac{e^{-t}}{-1}
}{0}{\infty}
- \int_0^\infty z t^{z-1} \frac{e^{-t}}{-1} dt
=
z \int_0^\infty t^{z-1} e^{-t} dt
=
z (z - 1)\int_0^\infty t^{z-2} e^{-t} dt
=
z (z - 1)(z - (3-1))\int_0^\infty t^{z-3} e^{-t} dt
=
z (z - 1) \cdots (z - (k-1))\int_0^\infty t^{z-k} e^{-t} dt
=
z (z - 1) \cdots (z - (k-1))\int_0^\infty t^{(z + 1 - k) - 1} e^{-t} dt,
\end{dmath}

or

\begin{dmath}\label{eqn:nonIntegralBinomialSeries:140}
\Gamma(z + 1) = z (z - 1) \cdots (z - (k-1)) \Gamma( z - (k-1) ).
\end{dmath}

Flipping this gives us a nice closed form expression for the products of a number of positive unit separated values

\begin{dmath}\label{eqn:nonIntegralBinomialSeries:160}
z (z - 1) \cdots (z - k)
=
\frac{\Gamma(z + 1)}{\Gamma( z - (k-1) )}.
\end{dmath}

\paragraph{Binomial coefficient for positive exponents and non-integer negative exponents}

Considering first positive exponents \(b\), we can now use this in our Taylor expansion \eqnref{eqn:nonIntegralBinomialSeries:60}

\begin{dmath}\label{eqn:nonIntegralBinomialSeries:180}
(a + x)^b
= \sum_{k = 0}^\infty \inv{k!} \frac{\Gamma(b+1)}{\Gamma(b - k + 1)}
a^{b-k} x^k
= \sum_{k = 0}^\infty \frac{\Gamma(b + 1)}{\Gamma(k + 1)\Gamma(b - k + 1)}
a^{b-k} x^k.
\end{dmath}

Observe that when \(b\) is a positive integer we have

\begin{equation}\label{eqn:nonIntegralBinomialSeries:200}
\frac{\Gamma(b + 1)}{\Gamma(k + 1)\Gamma(b - k + 1)}
=
\frac{b!}{k!(b-k)!}
= \binom{b}{k}.
\end{equation}

So for positive values of \(b\), integer or otherwise, and negative non-integer values, we see that is then very reasonable to define the \textAndIndex{binomial coefficient} explicitly in terms of the gamma function

\begin{equation}\label{eqn:nonIntegralBinomialSeries:220}
\binom{b}{k} \equiv
\frac{\Gamma(b + 1)}{\Gamma(k + 1)\Gamma(b - k + 1)}.
\end{equation}

If we do that, then the binomial expansion for non-integral values of \(b\) is simply

\begin{dmath}\label{eqn:nonIntegralBinomialSeries:240}
(a + x)^b = \sum_{k = 0}^\infty \binom{b}{k} a^{b-k} x^k.
\end{dmath}

\paragraph{Binomial coefficient for negative integer exponents}

Using the relation \eqnref{eqn:nonIntegralBinomialSeries:220} blindly leads to some trouble, since \(\Gamma(-\Abs{m})\) goes to infinity for integer values of \(m > 0\).  We have to modify the definition of the binomial coefficient for negative integer values.  Let's rewrite \eqnref{eqn:nonIntegralBinomialSeries:60} for negative integer values of \(b = -m\) as

\begin{dmath}\label{eqn:nonIntegralBinomialSeries:260}
(a + x)^{-m}
= \sum_{k = 0}^\infty \inv{k!}
(-m)(-m-1)\cdots (-m-(k-1)) a^{-m-k} x^k
= \sum_{k = 0}^\infty
\inv{k!}
(-1)^k
m(m+1)\cdots (m+(k-1)) a^{-m-k} x^k.
\end{dmath}

Let's also put the ratio of gamma functions relation of \eqnref{eqn:nonIntegralBinomialSeries:160}, in a slightly more general form.  For \(u, v > 0\), where \(u - v\) is an integer, we can write

\begin{dmath}\label{eqn:nonIntegralBinomialSeries:280}
u (u - 1) \cdots (v) = \frac{\Gamma(u + 1)}{\Gamma( v )}.
\end{dmath}

Our Taylor series takes the form

\begin{dmath}\label{eqn:nonIntegralBinomialSeries:300}
(a + x)^{-m}
= \sum_{k = 0}^\infty
\inv{k!}
(-1)^k
(m+k-1) (m+k-2) \cdots (m)
a^{-m-k} x^k
= \sum_{k = 0}^\infty
(-1)^k \frac{ \Gamma(m + k) }{\Gamma(m)\Gamma(k+1)}
a^{-m-k} x^k.
\end{dmath}

We can now define, for negative integers \(-m\)

\begin{equation}\label{eqn:nonIntegralBinomialSeries:320}
\binom{-m}{k}
\equiv
(-1)^k \frac{ \Gamma(m + k) }{ \Gamma(m)\Gamma(k+1) }.
\end{equation}

With such a definition, our Taylor series takes the tidy form
\begin{dmath}\label{eqn:nonIntegralBinomialSeries:300b}
(a + x)^{-m}
= \sum_{k = 0}^\infty \binom{-m}{k} a^{-m-k} x^k.
\end{dmath}

For negative integer values of \(b = -m\), this is now consistent with \eqnref{eqn:nonIntegralBinomialSeries:240}.

Observe that we can put \eqnref{eqn:nonIntegralBinomialSeries:320} into the standard binomial form with a bit of manipulation

\begin{dmath}\label{eqn:nonIntegralBinomialSeries:340}
\binom{-m}{k}
=
(-1)^k \frac{ \Gamma(m + k) }{ \Gamma(m)\Gamma(k+1) }
=
(-1)^k \frac{ (m + k -1)! }{ (m-1)! k! }
=
(-1)^k \frac{ m (m + k)! }{ (m + k) m! k! },
\end{dmath}

or
\begin{equation}\label{eqn:nonIntegralBinomialSeries:360}
\binom{-m}{k}
=
(-1)^k \frac{m}{m + k} \binom{m+k}{m}.
\end{equation}

%\paragraph{Negative non-integral binomial coefficients}
%
%TODO.  There will be some ugliness due to the changes of sign in the products \(b(b-1)\cdots (b -k + 1)\) since \(b\) and \(b -k + 1\) may have different sign.  A product of two ratios of gamma functions will be required to express this product, which will further complicate the definition of binomial coefficient.

%\EndArticle
