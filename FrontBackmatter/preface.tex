%
% Copyright � 2013 Peeter Joot.  All Rights Reserved.
% Licenced as described in the file LICENSE under the root directory of this GIT repository.
%

%
%
%\chapter{Preface}
% this suppresses an explicit chapter number for the preface.
\chapter*{Preface}%\normalsize
  \addcontentsline{toc}{chapter}{Preface}

This document is based on my lecture notes for the Winter 2013, University of Toronto Basic Statistical Mechanics course (PHY452H1S), taught by Prof.\ Arun Paramekanti.

\paragraph{Official course description:}

``Classical and quantum statistical mechanics of noninteracting systems; the statistical basis of thermodynamics; ensembles, partition function; thermodynamic equilibrium; stability and fluctuations; formulation of quantum statistics; theory of simple gases; ideal Bose and Fermi systems.''

\paragraph{This document contains:}

\begin{itemize}
\item Plain old lecture notes.   These mirror what was covered in class, possibly augmented with additional details.

\item Personal notes exploring details that were not clear to me from the lectures, or from the texts associated with the lecture material.

\item Assigned problems.  Like anything else take these as is.  I may or may not have gone back and corrected errors, and did not see the graded versions of the last two problem sets.

\item Some worked problems attempted as course prep, for fun, or for test preparation, or post test reflection.

\item Links to Mathematica workbooks associated with these notes.

\end{itemize}

\paragraph{On conventions}

These notes include worked problems from a few difference sources, which lead to some notational inconsistencies.  In particular the problems from \citep{kittel1980thermal} work with a fundamental temperature \(\tau = \kB T\) having dimensions of energy and a nondimensionalized entropy \(\sigma = S/\kB\).

I've attempted to use \(\kB\) consistently.  We used this or \(k\) in the class and our text \citep{pathriastatistical} uses \(k\).

For expressing the grand canonical partition function, I've settled on the \(\ZG\) notation from the final exam (not actually used at all in class).  This is less cumbersome than the \(\ZGorig\) that we used in class, and easier to interpret than the script Q used in the text \citep{pathriastatistical}.  That Q from the text looks like \(\ZGtext\), but with the circular part of the Q all detached and pushed up.  It took me a long time to even figure out what that symbol was supposed to be.

\paragraph{Thanks}

My thanks go to Professor Paramekanti, who knows his subject well, for teaching this course and for his excellent elaborations of a number of tricky seeming concepts.  I feel I learned a lot, despite also feeling like I've only scratched the surface of this subject.

Peeter Joot  \quad peeterjoot@protonmail.com
