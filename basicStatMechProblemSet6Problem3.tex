%
% Copyright � 2013 Peeter Joot.  All Rights Reserved.
% Licenced as described in the file LICENSE under the root directory of this GIT repository.
%
\makeoproblem{Nuclear matter}{basicStatMech:problemSet6:3}{\citep{huang2001introduction}, prob 9.2}{
%\makesubproblem{}{basicStatMech:problemSet6:3a}
%(3 points)
Consider a heavy nucleus of mass number \(A\).  i.e., having \(A\) total nucleons including neutrons and protons.  Assume that the number of neutrons and protons is equal, and recall that each of them has spin-\(1/2\) (so possessing two spin states).  Treating these nucleons as a free ideal Fermi gas of uniform density contained in a radius \(R = r_0 A^{1/3}\), where \(r_0 = 1.4 \times 10^{-13} \text{cm}\), calculate the Fermi energy and the average energy per nucleon in MeV.
} % makeoproblem

\makeanswer{basicStatMech:problemSet6:3}{

Our nucleon particle density is

\begin{dmath}\label{eqn:basicStatMechProblemSet6Problem3:20}
\rho
= \frac{N}{V}
= \frac{A}{\frac{4 \pi}{3} R^3}
= \frac{3 A}{4 \pi r_0^3 A}
= \frac{3}{4 \pi r_0^3}
= \frac{3}{4 \pi (1.4 \times 10^{-13} \text{cm})^3}
= 8.7 \times 10^{37} (\text{cm})^{-3}
= 8.7 \times 10^{43} (\text{m})^{-3}
\end{dmath}

With \(m\) for the mass of either the proton or the neutron, and \(\rho_m = \rho_p = \rho/2\), the Fermi energy for these particles is

\begin{dmath}\label{eqn:basicStatMechProblemSet6Problem3:40}
\epsilon_{\txtF} = \frac{\Hbar^2}{2m}
\lr{ \frac{6 \pi (\rho/2)}{2 S + 1}}
^{2/3},
\end{dmath}

With \(S = 1/2\), and \(2 S + 1 = 2(1/2) + 1 = 2\) for either the proton or the neutron, this is

\begin{dmath}\label{eqn:basicStatMechProblemSet6Problem3:60}
\epsilon_{\txtF} = \frac{\Hbar^2}{2 m}
\lr{ \frac{3 \pi^2 \rho}{2}}
^{2/3}.
\end{dmath}

\begin{dmath}\label{eqn:basicStatMechProblemSet6Problem3:80}
\begin{aligned}
\Hbar &= 1.05 \times 10^{-34} \,\text{\(m^2\) kg \(s^{-1}\)} \\
m &= 1.67 \times 10^{-27} \,\text{kg}
\end{aligned}.
\end{dmath}

This gives us
\begin{dmath}\label{eqn:basicStatMechProblemSet6Problem3:100}
\epsilon_{\txtF}
= \frac{(1.05 \times 10^{-34})^2}{2 \times 1.67 \times 10^{-27}}
\lr{ \frac{3 \pi^2 }{2}
\frac{8.7 \times 10^{43} }{2}
}
^{2/3}
\text{m}^4 \frac{\text{kg}^2}{s^2} \inv{\text{kg}} \inv{\text{m}^2}
=
3.9 \times 10^{-12} \,\text{J} \times
\lr{ 6.241509 \times 10^{12} \frac{\text{MeV}}{J}}
\approx
24 \text{MeV}
\end{dmath}

In lecture 16
%FIXME: \eqnref{eqn:basicStatMechLecture16:600}
% also Pathria 8.1.26
we found that the total average energy for a Fermion gas of \(N\) particles was
\begin{dmath}\label{eqn:basicStatMechProblemSet6Problem3:120}
E = \frac{3}{5} N \epsilon_{\txtF},
\end{dmath}

so the average energy per nucleon is approximately
\begin{dmath}\label{eqn:basicStatMechProblemSet6Problem3:140}
\frac{3}{5} \epsilon_{\txtF} \approx 15 \,\text{MeV}.
\end{dmath}
}
