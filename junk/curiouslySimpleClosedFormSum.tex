%
% Copyright � 2013 Peeter Joot.  All Rights Reserved.
% Licenced as described in the file LICENSE under the root directory of this GIT repository.
%
\input{../blogpost.tex}
\renewcommand{\basename}{curiouslySimpleClosedFormSum}
\renewcommand{\dirname}{notes/phy452/}
\newcommand{\keywords}{binomial coefficient, gamma function}

\input{../peeter_prologue_print2.tex}

\beginArtNoToc

\generatetitle{A curious Taylor series}
%\chapter{A curious Taylor series}
\label{chap:curiouslySimpleClosedFormSum}
%\section{Motivation}
%\section{Guts}

As part of a problem, I thought I needed the Taylor expansion of

\begin{dmath}\label{eqn:curiouslySimpleClosedFormSum:20}
(1 + \theta x)\lr{1 + \theta x/2}^{1/2}.
\end{dmath}

Take a look at the suprising simple final form form of the sum once evaluated.

\begin{equation}\label{eqn:curiouslySimpleClosedFormSum:40}
\begin{aligned}
(1 + \theta x)\lr{1 + \theta x/2}^{1/2}
&=
(1 + \theta x)
\sum_{s = 0}^\infty \binom{1/2}{s} \lr{\theta x/2}^s \\
&=
\sum_{s = 0}^\infty \binom{1/2}{s} \lr{\theta x/2}^s
+ 2 \frac{\theta x}{2}
\sum_{s = 0}^\infty \binom{1/2}{s} \lr{\theta x/2}^{s} \\
&=
1 + 
\sum_{s = 1}^\infty \binom{1/2}{s} \lr{\theta x/2}^s
+ 2
\sum_{s = 0}^\infty \binom{1/2}{s} \lr{\theta x/2}^{s + 1} \\
&= 
1
+
% t = s - 1
% t + 1 = s
\sum_{t = 0}^\infty \binom{1/2}{t + 1} \lr{\theta x/2}^{t + 1}
+
2
\sum_{s = 0}^\infty \binom{1/2}{s} \lr{\theta x/2}^{s + 1} \\
&=
1 +
\sum_{s = 0}^\infty 
\lr{ 
\binom{1/2}{s + 1} 
+
2
\binom{1/2}{s}
} \lr{\theta x/2}^{s + 1}.
\end{aligned}
\end{equation}

Our binomial coefficient has the usual definition

\begin{dmath}\label{eqn:curiouslySimpleClosedFormSum:60}
\binom{a}{b} 
=
\frac{a!}{b!(a-b)!},
\end{dmath}

however, because of the fractional powers here we require the gamma function generalization of the factorials

\begin{dmath}\label{eqn:curiouslySimpleClosedFormSum:80}
r! \equiv \Gamma(r+1),
\end{dmath}

for any \(r\) that is non-integral.

Now lets sum the binomial coefficients

\begin{equation}\label{eqn:curiouslySimpleClosedFormSum:100}
\begin{aligned}
\binom{1/2}{s + 1} 
+
2
\binom{1/2}{s}
&=
\frac{(1/2)!}{(s+1)!(1/2 -(s+1))!}
+ 2
\frac{(1/2)!}{s!(1/2 -s)!} \\
&=
\frac{(1/2)!}{s!(1/2 -s -1)!}
\lr{
\inv{s +1} + \frac{2}{1/2 - s}
} \\
&=
\cdots
\end{aligned}
\end{equation}

Never mind.  Doesn't look like it works after all?

%\EndArticle
\EndNoBibArticle
